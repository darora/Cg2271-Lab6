
\documentclass[runningheads,a4paper]{report}
\usepackage{listings}
\usepackage{color}
\lstset{breaklines=true}


% Title Page
\title{CG2271 RTOS Lab 6 --- Synchronization Mechanisms}
\author{Divyanshu Arora(U096857U), Omer Iqbal(MATRIC\_NO\_HERE)}


\begin{document}
\maketitle
\section*{Q1}
\subsection{Output}
\lstset{language=C, caption=Output of program from lab manual,
  label=Output, numbers=left, frame=shadowbox,
  rulesepcolor=\color{black}}
\begin{lstlisting}
PRINT: Current value of ctr is 0
PRINT: ctr is too small. Zzzz.. g'night!
ADD: New value of ctr is 1
ADD: New value of ctr is 2
ADD: New value of ctr is 3
ADD: New value of ctr is 4
ADD: New value of ctr is 5
ADD: New value of ctr is 6
ADD: New value of ctr is 7
ADD: New value of ctr is 8
ADD: New value of ctr is 9
ADD: New value of ctr is 10
ADD: Reached limit of 10! Waking up print thread
PRINT: ctr is now 10! Exiting thread.
\end{lstlisting}

\subsection{Description}

The program spawns two threads---\texttt{add} and \texttt{print}. It
uses a condition variable \texttt{ctr\_cond} alongwith the
accompanying mutex \texttt{ctr\_mutex} to ensure that the
\texttt{print} thread doesn't print and exit until the condition
$ctr>=MAX\_COUNT$ has been met.
This condition can only be met by execution of the other thread, which
increments the \texttt{ctr} variable if necessary, and signals
\texttt{ctr\_cond} to wake up one of the threads waiting on it when it
is done.

\section*{Q2}

From the \textit{POSIX Programmer's Manual} page on
\texttt{pthread\_cond\_wait}---
\begin{quotation}
[...]
These functions[\texttt{pthread\_cond\_wait} \& \texttt{pthread\_cond\_timedwait}] atomically release \texttt{mutex}
  and cause the calling thread to block on the condition variable
  \texttt{cond}; atomically here means ``atomically with respect to access by another thread to the mutex and  then  the  condition
       variable''.  That  is, if another thread is able to acquire the mutex after the about-to-block thread has released it, then a subsequent call to \texttt{pthread\_cond\_broadcast()} or \texttt{pthread\_cond\_signal()} in that thread shall behave as
       if it were issued after the about-to-block thread has blocked. [...]
\end{quotation}

Since the \texttt{ctr\_mutex} variable is released by the
\texttt{pthread\_cont\_wait} call, there is no possibility for
deadlock.

\section*{Q3}

The first thread to call \texttt{sem\_wait} doesn't have to block,
because the call to \texttt{sem\_init} is passed an initial value of
\texttt{1}. After the first thread decrements the value, it
will still be non-zero, and therefore not block.

\section*{Q4}

One thread is woken up each time the main thread calls
\texttt{sem\_post}.

\section*{Q5}

\begin{itemize}
  \item Mutex provides mutual exclusion to a resource, i.e. only one
    thread can access the resource at a time. A semaphore can provide
    access to a user-defined number of threads.
  \item A semaphore can be shared between processes, whereas a mutex
    can't.
\end{itemize}

\section*{Q6}

\lstset{caption=git-diff of changes made to ensure that print executes
  before add}
\begin{lstlisting}
--- a/lab6_1.c
+++ b/lab6_1.c
@@ -29,6 +29,8 @@ void *print(void *p)
   if(ctr < MAX_COUNT)
     {
       printf("PRINT: ctr is too small. Zzzz.. g'night!\n");
+      pthread_cond_signal(&ctr_cond); //wake up main, and allow it to
+                                      //create add thread
       pthread_cond_wait(&ctr_cond, &ctr_mutex);
       printf("PRINT: ctr is now %d! Exiting thread.\n", ctr);
     }
@@ -41,8 +43,13 @@ int main()
   ctr=0;
   pthread_mutex_init(&ctr_mutex, NULL);
   pthread_cond_init(&ctr_cond, NULL);
-  pthread_create(&threads[0], NULL, add, NULL);
   pthread_create(&threads[1], NULL, print, NULL);
+
+  pthread_mutex_lock(&ctr_mutex);
+  pthread_cond_wait(&ctr_cond, &ctr_mutex);
+  pthread_create(&threads[0], NULL, add, NULL);
+  pthread_mutex_unlock(&ctr_mutex);
+
   pthread_join(threads[1], NULL);
   pthread_mutex_destroy(&ctr_mutex);
   pthread_cond_destroy(&ctr_cond);
\end{lstlisting}

To summarize, the main thread spawns the \texttt{print} thread first,
and then calls \texttt{pthread\_cond\_wait} on the \texttt{ctr\_cond}
variable that was already being used to synchronize \texttt{print} and
\texttt{add}.

The \texttt{print} thread, in turn, calls
\texttt{pthread\_cond\_signal} to wake up the \texttt{main} thread
before calling \texttt{pthread\_cond\_wait}. This ensures that the
\texttt{main} thread can go on creating the \texttt{add} thread as
before and execute it.

\section*{Q7}

Expected answer is $49*(49+1)/2$, or $1225$.

\section*{Q8}

The program doesn't give the correct answer.

The library specifications mention that the \texttt{pq\_get} call
should block if the queue is empty. However, the given skeleton
implementation returns a designated error code (\textit{-999}). The
test suite, of course, follows the specifications, and instead treats
\textit{-999} as just any other number from the queue.

\section*{Q9}

Changes made are ---

\begin{itemize}
  \item Added \texttt{mutex}, \texttt{\texttt{empty}} cond\_var to the \texttt{pq\_t} struct.
  \item Initialize aforementioned variables in the \texttt{pq\_create}
    function, and destroy them in \texttt{pq\_destroy}
  \item Use the mutex \& condition variable to block the
    \texttt{pq\_put} and \texttt{pq\_get} functions if necessary, and
    to call \texttt{pthread\_cond\_broadcast} to unblock as appropriate.
\end{itemize}

A diff of the exact changes made is attached---
\lstset{caption=Changes made to the queue library to follow the specs
  provided}
\lstinputlisting{./q9.diff}

\section*{Q10}

Changes made are ---
\begin{itemize}
  \item Included \texttt{semaphore.h} and added a global
    \texttt{sem\_t} variable, which is given an initial value of $6$,
    i.e. the number of readers + the number of writers.
  \item The \texttt{main} thread calls \texttt{sem\_wait} on the
    semaphore before it spawns off a new thread.
  \item Each of the reader \& writer threads call \texttt{sem\_post}
    on  the semaphore just before exiting.
  \item The \texttt{main} thread calls \texttt{sem\_wait} once more
    just before calling \texttt{pq\_destroy} on the queue.
\end{itemize}

A diff of the exact changes is attached---

\lstset{caption=Changes made to enable calling \texttt{pq\_destroy} on
  the queue.}
\lstinputlisting{./q10.diff}

\end{document}
