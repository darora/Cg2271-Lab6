
\documentclass[runningheads,a4paper]{report}
\usepackage{listings}
\usepackage{color}
\lstset{breaklines=true}


% Title Page
\title{CG2271 RTOS Lab 6 --- Synchronization Mechanisms}
\author{Divyanshu Arora(U096857U), Omer Iqbal(MATRIC\_NO\_HERE)}


\begin{document}
\maketitle
\section*{Q1}
\subsection*{Output}
\lstset{language=C, caption=Output of program from lab manual,
  label=Output, numbers=left, frame=shadowbox,
  rulesepcolor=\color{black}}
\begin{lstlisting}
PRINT: Current value of ctr is 0
PRINT: ctr is too small. Zzzz.. g'night!
ADD: New value of ctr is 1
ADD: New value of ctr is 2
ADD: New value of ctr is 3
ADD: New value of ctr is 4
ADD: New value of ctr is 5
ADD: New value of ctr is 6
ADD: New value of ctr is 7
ADD: New value of ctr is 8
ADD: New value of ctr is 9
ADD: New value of ctr is 10
ADD: Reached limit of 10! Waking up print thread
PRINT: ctr is now 10! Exiting thread.
\end{lstlisting}

\subsection*{Description}

The program spawns two threads---\texttt{add} and \texttt{print}. It
uses a condition variable \texttt{ctr\_cond} alongwith the
accompanying mutex \texttt{ctr\_mutex} to ensure that the
\texttt{print} thread doesn't print and exit until the condition
$ctr>=MAX\_COUNT$ has been met.
This condition can only be met by execution of the other thread, which
increments the \texttt{ctr} variable if necessary, and signals
\texttt{ctr\_cond} to wake up one of the threads waiting on it when it
is done.







\end{document}
